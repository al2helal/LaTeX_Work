%text taken from Mawlana Murshidul Islam, Jamia Arabia Haji Unus(Kawmi Madrasha)
% ornamented and decorated by Md.Al-Helal
\documentclass[12pt]{scrartcl}
\usepackage{polyglossia}
\usepackage{booktabs}
\renewcommand{\arraystretch}{1.3}
\setmainlanguage[numerals=Devanagari]{bengali}
\setmainlanguage{bengali}
\setotherlanguage{english}
\newfontfamily\bengalifont[Script=Bengali]{NikoshLightBAN}
\usepackage[a4paper,left=1cm,right=1cm,top=1cm,bottom=1cm]{geometry}

\usepackage[utf8]{inputenc} 
\usepackage[T1]{fontenc} 
\usepackage[dvipsnames]{xcolor}
\usepackage[object=vectorian]{pgfornament} % also loads tikz

\tikzset{pgfornamentstyle/.style={draw = Periwinkle,
                                  fill = SpringGreen}}   

\usetikzlibrary{
  positioning, % for left=of, above=of etc.
  calc % for let syntax used in second example
}

\begin{document}  

\begin{center}   
%\begin{tikzpicture}[
%  every node/.append style={inner sep=0},
%  node distance=5mm
%]
%   \node [Maroon] (text) {
%   \begin{tabular}{ll}
%     শিশুকে \\
%     ভুল শেখাবেন না & সঠিক শেখাবেন।\\
%     লোভ দেখাবেন না & পুরস্কার দেবেন \\
%     নিরুৎসাহিত করবেন না & উৎসাহিত করবেন\\
%     বকুনি দেবেন না&উপেদশ দেবেন \\
%     বেয়াদব বলবেন না&আদব শেখাবেন \\
%     নিঃসঙ্গ  রাখবেন না&সৎসঙ্গ দেবেন \\
%     ভয় দেখাবেন না&সাহস দেবেন \\
%     লজ্জা দেবেন না& শিখিয়ে দেবেন \\
%     ধমক দেবেন না&বুঝতে দেবেন \\
%     মিথ্যা বলবেন না&সত্য বলবেন \\
%     বেত্রাঘাত করবেন না &স্নেহ করবেন \\
%     নিন্দা করেবন না&ভালবাসবেন \\
%     অবেহলা করবেন না&গুরুত্ব দেবেন \\
%     মাওলানা মুরশিদুল আলম & মুহতামিম\\ 
%     জামিয়া আরাবিয়া হাজী ইউনুছ (কওমী মাদ্রাসা)
%    \end{tabular}
%    };
%
%  \node [anchor=north] (below) at (text.south) {\pgfornament[width=5cm,symmetry=c]{69}};
%  \node [anchor=south] (above) at (text.north) {\pgfornament[width=5cm]{69}};
%
%   \node [rotate=-90, left=of text, anchor=north] (left)  {\pgfornament[width=6cm]{46}};
%   \node [rotate=90, right=of text, anchor=north] (right) {\pgfornament[width=6cm]{46}};
%
%   \node [above=of above] (top)    {\pgfornament[width=6cm]{71}};
%   \node [below=of below] (bottom) {\pgfornament[width=6cm,symmetry=h]{71}};
%
%   \node [anchor=north west] at (top.north -| left.south)  {\pgfornament[width=2cm]{63}};
%   \node [anchor=north east] at (top.north -| right.south) {\pgfornament[width=2cm,symmetry=v]{63}};
%   \node [anchor=south west] at (bottom.south -| left.south) {\pgfornament[width=2cm,symmetry=h]{63}};
%   \node [anchor=south east] at (bottom.south -| right.south) {\pgfornament[width=2cm,symmetry=c]{63}};
%
%   % draw frame
%  \draw [Blue] (current bounding box.south west) rectangle (current bounding box.north east);
%
%\end{tikzpicture} 
%

\begin{tikzpicture}[
  every node/.append style={inner sep=0},
  node distance=5mm
]
   \node [Maroon] (text) {
   \begin{tabular}{ll}
     \hspace{4.7cm} {\LARGE শিশুকে }\\
\toprule
     ভুল শেখাবেন না & সঠিক শেখাবেন।\\
     লোভ দেখাবেন না &পুরস্কার দেবেন \\
     নিরুৎসাহিত করবেন না & উৎসাহিত করবেন \\
     বকুনি দেবেন না & উপেদশ দেবেন \\
     বেয়াদব বলবেন না & আদব শেখাবেন \\
     নিঃসঙ্গ  রাখবেন না & সৎসঙ্গ দেবেন \\
     ভয় দেখাবেন না & সাহস দেবেন \\
     লজ্জা দেবেন না & শিখিয়ে দেবেন \\
     ধমক দেবেন না & বুঝতে দেবেন \\
     মিথ্যা বলবেন না & সত্য বলবেন \\
     বেত্রাঘাত করবেন না & স্নেহ করবেন \\
     নিন্দা করবেন না & ভালবাসবেন \\
     অবেহলা করবেন না & গুরুত্ব দেবেন \\
\bottomrule
     \hspace{2.8cm}মাওলানা মুরশিদুল আলম (মুহতামিম)\\ 
     \hspace{2.8cm}জামিয়া আরাবিয়া হাজী ইউনুছ (কওমী মাদ্রাসা)
    \end{tabular}
    };



  \path 
    let
    \p1=(text.south west),
    \p2=(text.north east),
    \n1={\x2-\x1}, % width of text node
    \n2={\y2-\y1}  % height of text node
    in
% in all of the below some fraction of \n1 or \n2 is used to define the width of the ornaments
% set width of these ornaments to half the text node's width
   node [anchor=north] (below) at (text.south) {\pgfornament[width=0.5*\n1,symmetry=c]{69}}
   node [anchor=south] (above) at (text.north) {\pgfornament[width=0.5*\n1]{69}}
% use \n2 for ornament width here
   node [rotate=-90, left=of text, anchor=north] (left)  {\pgfornament[width=\n2]{46}}
   node [rotate=90, right=of text, anchor=north] (right) {\pgfornament[width=\n2]{46}}
% and \n1 here
   node [above=of above] (top)    {\pgfornament[width=\n1]{71}}
   node [below=of below] (bottom) {\pgfornament[width=\n1,symmetry=h]{71}}
%   node [anchor=north] (top) at (text.north)   {\pgfornament[width=\n1]{71}}
%   node [anchor=south] (bottom) at (text.south) {\pgfornament[width=\n1,symmetry=h]{71}}
% and 25% of width for the corner ornaments
   node [anchor=north west] at (top.north -| left.south)  {\pgfornament[width=0.25*\n1]{63}}
   node [anchor=north east] at (top.north -| right.south) {\pgfornament[width=0.25*\n1,symmetry=v]{63}}
   node [anchor=south west] at (bottom.south -| left.south) {\pgfornament[width=0.25*\n1,symmetry=h]{63}}
   node [anchor=south east] at (bottom.south -| right.south) {\pgfornament[width=0.25*\n1,symmetry=c]{63}}; % <- note the \path doesn't end until here

   % draw frame
  \draw [Blue] (current bounding box.south west) rectangle (current bounding box.north east);

\end{tikzpicture} 
\end{center}
\end{document} 
