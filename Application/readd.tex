\documentclass[a4paper]{article}
\usepackage{fullpage}
\usepackage{polyglossia,anyfontsize}
\setmainlanguage[numerals=Devanagari]{bengali}
\setmainlanguage{bengali}
\setotherlanguage{english}
\newfontfamily\bengalifont[Script=Bengali]{NikoshLightBAN}
\begin{document}
\fontsize{17}{20}
\selectfont
\vfill
\noindent
২৯ সেপ্টেম্বর, ২০২১\\ 

\noindent
বরাবর,\\ 
প্রভোস্ট,\\
ড. মুহম্মদ শহীদুল্লাহ হল,\\
ঢাকা বিশ্ববিদ্যালয়,\\
ঢাকা - ১০০০\\

\noindent
মাধ্যম: চেয়ারম্যান, কম্পিউটার বিজ্ঞান ও প্রকৌশল বিভাগ\\

\noindent
বিষয়: পুনঃভর্তির মাধ্যমে পূর্বের থিওরি পরীক্ষার নম্বর অপরিবর্তিত রেখে থিসিস জমাদানের  সুযোগের জন্য আবেদন।\\

\noindent
জনাব,
সবিনয় নিবেদন এই যে, আমি মোঃ আল - হেলাল, অনার্স ভর্তি সেশন: ২০১৪-১৫, রেজি: নম্বর: ২০১৪-১১৬-৬৫৪, মাস্টার্স সেশন: ২০১৮-১৯, আমি আপনার হলের  একজন আবাসিক ছাত্র। ইতিমধ্যে আমার মাস্টার্স এর দুই সেমিস্টার শেষ হয়েছে। এখন আমার থিসিস সেমিস্টার বাকি আছে। আমার থিসিসের কাজ সম্পন্ন না হওয়ায় আমি গত সেমিস্টারের ডিফেন্সে অংশগ্রহণ করতে পারি নি। আমার মাস্টার্স এর লাস্ট সেমিস্টার  \selectlanguage{english} MS Fall Semester 2019
\selectlanguage{bengali}

\noindent
এমতাবস্তায়, আমি আগামী ২০২০-২১ সেশনের থিসিস সেমিস্টারে অংগ্রহণ করতে আগ্রহী। সুতরাং, জনাবের নিকট আবেদন যাতে করে আমাকে পুনঃভর্তির মাধ্যমে পূর্বের থিওরি পরীক্ষার নম্বর অপরিবর্তিত রেখে থিসিস জমাদানের  সুযোগ করে দেওয়া হয়।\\



\noindent
বিনীত নিবেদক-\\
মোঃ আল - হেলাল\\
রেজি: নম্বর: ২০১৪-১১৬-৬৫৪\\
মোবাইল: \selectlanguage{english} 01515 - 611969
\end{document}
