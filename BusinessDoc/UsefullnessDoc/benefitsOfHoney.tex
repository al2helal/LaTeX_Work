\documentclass[a4paper,12pt]{article}
\usepackage[multiple]{footmisc}
%\usepackage{multido,graphicx,tikz}
%\usepackage{booktabs}
%\usepackage{enumitem,anyfontsize}
%\usepackage[a4paper]{geometry}
\usepackage{polyglossia}
\setmainlanguage[numerals=Devanagari]{bengali}
%\setmainlanguage{bengali}
\setotherlanguage{english}
%\newfontfamily\englishfont[Scale=MatchLowercase]{Linux Biolinum O}
\newfontfamily\bengalifont[Script=Bengali]{NikoshLightBAN}
\begin{document}
\title{মধুর বিস্ময়কর উপকারিতা}
\author{মোঃ আল - হেলাল\\ কম্পিউটার বিজ্ঞান ও প্রকৌশল\\ ঢাকা বিশ্ববিদ্যালয়}
\maketitle
প্রাচীন কাল থেকেই মধু খাদ্য ও ঔষধ উভয় হিসাবে ব্যবহার হয়ে আসছে।  ইহা উপকারী উদ্ভিজ্জ যৌগ সমৃদ্ধ এবং অনেক স্বাস্থ্যগত সুবিধা প্রদান করে। মধু খুব স্বাস্থ্যকর, বিশেষভাবে  যখন পরিশোধিত চিনির পরিবর্তে ইহা ব্যবহার করা হয়, যা ১০০ \% ক্যালোরি সমৃদ্ধ।

এখানে মধুর শীর্ষ ১০ স্বাস্থ্য সুবিধা দেওয়া হল -
\section{মধুর পুষ্টিগুণ}
মধু মৌমাছি দ্বারা তৈরি একটি মিষ্টি, গাঢ় তরল। মৌমাছি তাদের আশপাশের চিনি সমৃদ্ধ ফুল থেকে মধু সংগ্রহ করে। একবার মৌচাকে মধু রাখার পর তারা বারবার ঐ মধু খায় এবং পাকস্থলী থেকে মুখে উগরে আনার পর পুনরায় মৌচাকে রাখে। সর্বশেষ পণ্যটি হচ্ছে তরল মধু, যা মৌমাছির সংরক্ষিত খাদ্য হিসাবে সঞ্চিত থাকে।  মৌমাছি কোন কোন ফুলে ঘুরেছে তার উপর ভিত্তি করে মধুর গন্ধ, রঙ ও স্বাদ বিভিন্ন ধরনের হয়।
পুষ্টিগতভাবে, ১ টেবিল-চামচ (২১ গ্রাম) মধুতে ৬৪ ক্যালোরি এবং ১৭ গ্রাম চিনিসহ ফ্রুক্টোজ, গ্লুকোজ, মল্টোজ এবং সুক্রোজ রয়েছে। এটাতে কার্যত কোন ফাইবার, চর্বি বা প্রোটিন নেই। এটিতে ভিটামিন এবং খনিজ কম থাকে কিন্তু কিছু উদ্ভিজ্জ যৌগ উচ্চ হারে থাকে। মধু উজ্জ্বল হয় তার বায়োঅ্যাক্টিভ উদ্ভিজ্জ যৌগ এবং অ্যান্টিঅক্সিডেন্টসমূহের কারণে। গাঢ় ধরনের মধুতে এই সমস্ত যৌগ উচ্চ মাত্রায় থাকে।\footnote{\selectlanguage{english}https://www.ncbi.nlm.nih.gov/pmc/articles/PMC3583289/}\footnote{\selectlanguage{english}https://www.sciencedirect.com/science/article/pii/S0308814608013733}
\section{উচ্চগুণ সম্পন্ন মধু অ্যান্টিঅক্সিডেন্ট সমৃদ্ধ}
উচ্চ মানের মধুতে অনেক গুরুত্বপূর্ণ অ্যান্টিঅক্সিডেন্টসমূহ রয়েছে। এতে ফ্ল্যাভোনিয়েডস এর মত জৈব অ্যাসিড এবং ফেনোলিক যৌগ আছে। বিজ্ঞানীরা বিশ্বাস করেন যে, এই যৌগগুলির সমন্বয় মধুকে তার অ্যান্টিঅক্সিডেন্ট শক্তি দেয়।\footnote{\selectlanguage{english}https://www.ncbi.nlm.nih.gov/pubmed/12358452}
%মজার ব্যাপার যে, দুই গবেষণায় দেখানো হয়েছে যে বাজরা ({\selectlanguage{english}buckwheat}) ফুলের মধু আপনার রক্তের অ্যান্টিঅক্সিডেন্ট মান বাড়ায়।
অ্যান্টিঅক্সিডেন্টগুলি হার্ট অ্যাটাক, স্ট্রোক এবং কয়েক ধরণের ক্যান্সারের ঝুঁকি কমায়। অ্যান্টিঅক্সিডেন্টগুলি চোখের স্বাস্থ্য উন্নতিও করতে পারে। অ্যান্টিঅক্সিডেন্টগুলি অনেক সময় বিভিন্ন অণুর অক্সিডেসনকে বাধা দেয় এবং কিছু ক্ষেত্রে প্রতিরোধ করে। শরীরে ফ্রি রেডিকেল বা মুক্ত মূলক যত বাড়ে তত কঠিন রোগ হওয়ার সম্ভবনা বাড়ে।  অ্যান্টিঅক্সিডেন্টগুলি এইসব ফ্রি রেডিকেলকে নিষ্ক্রিয় করে দেয়। মানব দেহ স্বাভাবিকভাবেই  ফ্রি রেডিকেল বা মুক্ত মূলক  তৈরি করে এবং তাদের ক্ষতিকর প্রভাবগুলি প্রতিরোধ করার জন্য সাথে সাথে অ্যান্টিঅক্সিডেন্টগুলিও তৈরি করে। কিন্তু বেশিরভাগ ক্ষেত্রে, ফ্রি রেডিকেল বা মুক্ত মূলক গুলি স্বাভাবিকভাবে  তৈরি হওয়া অ্যান্টিঅক্সিডেন্টগুলির চেয়ে অনেক বেশি হয়। তাই, ভারসাম্য বজায় রাখার জন্য, বাহিরের থেকে অ্যান্টিঅক্সিডেন্টগুলির ক্রমাগত সরবরাহ প্রয়োজন। অ্যান্টিঅক্সিডেন্টগুলি রক্ত ​​প্রবাহ থেকে ফ্রি রেডিকেল বা মুক্ত মূলকগুলিকে নিরপেক্ষ করে এবং অপসারণ করে শরীরকে উপকৃত করে। যখন ত্বক অতিবেগুনী আলোর উচ্চ মাত্রায় উন্মুক্ত থাকে, তখনঅক্সিজেনের বিভিন্ন ধরনের প্রতিক্রিয়াশীল প্রজাতির ({\selectlanguage{english}reactive species of oxygen})(যেমন- মুক্ত অক্সিজেন, সুপারক্সাইড রেডিকেল এবং পেরক্সাইড রেডিকেল) দ্বারা ত্বকে  ফটো-অক্সিডেটিভ ক্ষতি হয়। প্রতিক্রিয়াশীল অক্সিজেনের এই প্রজাতি গুলো ত্বকের সেলুলার লিপিড, প্রোটিন এবং ডিএনএ ({\selectlanguage{english}DNA}) এর ক্ষতি করে এবং চামড়ার অকাল বার্ধক্য, ত্বকের ক্যান্সার ইত্যাদির কারণ হয়।
অস্ট্যাক্স্থিন ({\selectlanguage{english}Astaxanthin}), ভিটামিন {\selectlanguage{english}E} এর সাথে মিলিত বিটা-ক্যারোটিন, অক্সিজেনের প্রতিক্রিয়াশীল প্রজাতির থেকে ত্বকের সুরক্ষার জন্য সবচেয়ে শক্তিশালী অ্যান্টিঅক্সিডেন্ট হিসাবে কাজ করে।
\selectlanguage{bengali}
\end{document}
