\documentclass[usenames,dvipsnames]{beamer}
\beamertemplatenavigationsymbolsempty
\usepackage{tikz,multicol,forest}
\usepackage{filecontents}
\usepackage{amsmath,amssymb,amsfonts}
\usepackage{algorithm}
\usetikzlibrary{arrows.meta}
\usepackage{graphicx}
\usepackage{textcomp}
\usepackage{xcolor,booktabs,multirow,tabularx}
\usetheme{Madrid}
\logo{%
  \vspace{-0.3cm} \includegraphics[width=1cm,height=1cm,keepaspectratio]{Image/DUlogo.png}%
  \hspace{\dimexpr\paperwidth-2cm-5pt}%
  \includegraphics[width=1cm,height=1cm,keepaspectratio]{Image/cseduT.png}%
  %\includegraphics[width=1cm,height=1cm,keepaspectratio]{example-image-a}%
}
\begin{document}
  \title{Bangladesh Bank}
  \author[Md.Al-Helal]{Md. Al-Helal}
\institute[CSEDU]{Computer Science \& Engineering\\University of Dhaka}
\date{February 01, 2019}
\makeatletter
\setbeamertemplate{footline}{%
  \leavevmode%
  \hbox{%
    \begin{beamercolorbox}[wd=.22\paperwidth,ht=2.25ex,dp=1ex,center]{author in head/foot}%
      \usebeamerfont{author in head/foot}\insertshortauthor\expandafter\ifblank\expandafter{\beamer@shortinstitute}{}{~~(\insertshortinstitute)}
    \end{beamercolorbox}%
    \begin{beamercolorbox}[wd=.73\paperwidth,ht=2.25ex,dp=1ex,center]{title in head/foot}%
      \usebeamerfont{title in head/foot}\insertshorttitle
    \end{beamercolorbox}%
  }%
  \begin{beamercolorbox}[wd=.05\paperwidth,ht=2.25ex,dp=1ex,right]{date in head/foot}%
    \usebeamerfont{date in head/foot}%
    \usebeamertemplate{page number in head/foot}%
    \hspace*{1ex} 
  \end{beamercolorbox}
  \vskip0pt%
}
\makeatother
\begin{frame}
  \maketitle
\end{frame}

\begin{frame}
\frametitle{Contents}
\tableofcontents
\end{frame}

\section{Background}
\begin{frame}
\frametitle{Background}
Bangladesh Bank is the central bank of Bangladesh and member of the \textbf{Asian Clearing Union}.\\
The bank is active in developing \textbf{green banking} and financial inclusion policy and is an important member of the Alliance for Financial Inclusion. Bangladesh Financial Intelligence Unit (\textbf{BFIU}) got the membership of \textbf{Egmont Group}.
\end{frame}

\begin{frame}
 \frametitle{Background}
 \begin{itemize}
  \item First central bank in the world to introduce a \textbf{dedicated hotline (16236)} for the populace to complain any banking related problem.
\item First central bank in the world to issue a \textbf{Green Banking Policy}.
\item To acknowledge this contribution, then-governor \textbf{Dr. Atiur Rahman} was given the title \textbf{Green Governor} at the 2012 United Nations Climate Change Conference, held at the Qatar National Convention Center in Doha.
  \end{itemize}
  \end{frame}

\section{History}
\begin{frame}[allowframebreaks]
\frametitle{History}
\begin{itemize}
\item After the Independence War and the eventual independence of Bangladesh, the Government of Bangladesh reorganized the \textbf{Dhaka branch} of the \textbf{State Bank of Pakistan} as the central bank of the country, naming it \textbf{Bangladesh Bank}.
\item This \textbf{re-organization} was done pursuant to \textbf{Bangladesh Bank Order, 1972}, and the Bangladesh Bank came into existence.
\item The \textbf{1972} Mujib government pursued a pro-socialist agenda. In 1972, the government decided to \textbf{nationalize} all banks to \textbf{channel funds} to the public sector and to prioritize credit to those sectors that sought to \textbf{reconstruct} the \textbf{war-torn country} – mainly \textbf{industry} and \textbf{agriculture}.
\item In \textbf{1982}, the first reform program was initiated, wherein the government \textbf{denationalised two of the six} nationalised commercial banks and permitted private local banks to compete in the banking sector.
\item In \textbf{1986}, a \textbf{National Commission on Money, Banking and Credit} was appointed to deal with the problems of the banking sector, and a number of steps were taken for the recovery targets for the nationalised commercial banks and development financial institutions and prohibiting \textbf{defaulters} from getting new loans.
\end{itemize}
\end{frame}

\section{Branch offices}
\begin{frame}
\frametitle{Branch Officies}
\begin{multicols}{2}
\begin{itemize}
\item Motijheel
\item Sadarghat
\item Bogura
\item Chattogram
\item Rajshahi
\item Barishal
\item Khulna
\item Sylhet
\item Rangpur
\item Mymensingh
\end{itemize}
\end{multicols}

\end{frame}

\section{Functions}
\begin{frame}
\frametitle{Functions}
\begin{itemize}
\item Formulation and implementation of monetary and credit policies.
\item Regulation and supervision of banks and non-bank financial institutions, promotion and development of domestic financial markets.
\item Management of the country's international reserves.
\item Issuance of currency notes.
\item Regulation and supervision of the payment system.
\item Acting as banker to the government .
\item Money laundering prevention.
\item Collection and furnishing of credit information.
\item Implementation of the Foreign Exchange Regulation Act.
\item Managing a deposit insurance scheme .
\end{itemize}
\end{frame}

\section{Organisation}
\begin{frame}
\frametitle{Organisation}
\begin{itemize}
\item The bank's highest official is the governor. His seat is in Motijheel, Dhaka. The governor chairs the board of directors.
\item The \textbf{executive staff} is responsible for the bank's day-to-day affairs.
\item \textbf{Departments:} \begin{itemize}
                             \item Debt Management
                             \item Law and so on
                            \end{itemize}
Each department is headed by one or more general managers.
\item The Bank has \textbf{10 physical branches}. Each is headed by a \textbf{general manager}. Headquarters are located in the \textbf{Bangladesh Bank Building} in Motijheel, which has two general managers. 
\end{itemize}
\end{frame}

\section{Hierarchy}
\begin{frame}
\frametitle{Hierarchy}
\begin{forest}
  for tree={
    align=center,
    parent anchor=south,
    child anchor=north,
    font=\sffamily,
    edge={thick, -{Stealth[]}},
    l sep+=10pt,
    edge path={
      \noexpand\path [draw, \forestoption{edge}] (!u.parent anchor) -- +(0,-10pt) -| (.child anchor)\forestoption{edge label};
    },
    if level=0{
      inner xsep=0pt,
      tikz={\draw [thick] (.south east) -- (.south west);}
    }{}
  }
  [Executive Staff
    [Governors
    [Executive Directors
    [General Manager]]
    [Economic Advisor]]
    [Three Deputy Governors]
  ]
\end{forest}
\begin{itemize}
\item The \textbf{Executive Staff} is responsible for \textbf{daily affairs}.
\item Although the \textbf{general managers} come under the \textbf{executive directors} they are not part of the executive staff.
\item The \textbf{board of directors} consists of the bank's \textbf{governor} and \textbf{eight other members} responsible for the policies undertaken by the bank.
\end{itemize}
\end{frame}

\section{Publications}
\begin{frame}
\frametitle{Publications}
\begin{itemize}
\item Annual Report
\item Bangladesh Bank Quarterly
\item Monetary Policy Review
\item CSR Initiatives in Banks
\item BBTA Journal : Thoughts on Banking and Finance
\item Annual Report on Green Banking
\item Import Payments
\item Financial Stability Assessment Report
\end{itemize}
\end{frame}

\begin{frame}{\phantom{}}
 % \color{Brown}
  \color{Sepia}
  \centering \Huge\textbf{Thank You}
\end{frame}
\end{document}
