\documentclass{article}
\begin{document}
\title{Resource Allocation in Cloud Radio Access Network}
\author{Md.Al-Helal (SH-51) \\ Jobayed Ullah (EK-107)} 
\date{\today}
\maketitle
\section{Introduction}
Job opportunity is not enough for the graduate students in our country. Therefore, the job provider try to take best one. Detecting the factors, those make one a best one in his sector is important. Our goal is finding the factors affecting the job opportunity  so that student can give attention to improve those.

\subsection{Problem Definition}
\subsubsection{Motivation}
Nowadays, some of CSE student depressed about their job opportunity. They have no statistics of their seniors and they don't know which factors are responsible for getting a better job. Consequently, they complete their graduation somehow and make intention to take MBA, CA even BCS although he has studied in a productive and technical subject. Some are also involved in business. Therefore, we will try to investigate the factors that affects the chance of getting better jobs based on the data produced by the CSE graduate students and build a predictive model. The estimated model will be able to predict the chance of getting better jobs on the basis of background characteristics of an under-graduate student. As a result, the   under-graduate student may be conscious about their lacking factors and may try to improve themselves.  The prediction will continually motivate the students through identifying the strong and weak points of the students also.

\subsection{Objective}
We will collect alumni's data on CGPA, study hour per day, attendance, communication skill, performance in contest, performance in extra curricular activities, political background, assignment and homework performance, study hour in science library, connection to elder brothers and teachers, religious activity, father's status, marital status, having friends, tution, job, salary   etc. providing online form. Obviously, we will protect their privacy while collecting data. Then we will analyze those data using different statistical and mathematical methodologies to establish a model. And analyzing those data we will predict on undergraduate student's data. An undergraduate student will check his probability of getting good job, salary from  his given data. He will understand his present situation with respect to other students with whom he is going to fight in future for a job. He will realize how much improvement  he needs to get a better job.\\
For this course as we have not enough time so we will work on CSE students of our university. But, in future we will generalize the model to predict for a student of any department.


\section{Related Work}
The main data source means the population is the alumni of CSE dept.


\subsection{Problem those paper address}
Md. Riaz Uddin one of our elder brother, MS students of CSE department and who worked on this sector, we talked with him and knew how he had done. He give us some idea and technique to collect data easily.\\We also talked with Momenul Haque Mondol, Senior Research Assistant, ICDDR,B who studied from ISRT of University of Dhaka. He advised us how to choose parameter and prepare questionnaire so that people give us data without hesitation.\\
We will take help for our project from these persons and obviously from you.

\subsection{Methods and positive aspects of those papers}
The main challenge of the project is sampling and data collection. A good sample is very important to get a good output. It is very hard to meet manually with all the graduate persons since most of them are busy. Their locations are different and we don't know all of them. So, data collection is mostly online based. A lot of people do not feel interested to fill up an online form.


\subsection{Challenges remain on those papers}
The data collection of our project is costly. For current working we can bear the cost. But, in future if our department or any organization will support us we will be able to continue easily to generalize the model for all departments not only CSE.

\end{document}
