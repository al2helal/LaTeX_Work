\documentclass[12pt]{article}
\usepackage{forloop,pbox}
\usepackage{polyglossia}
\usepackage{booktabs}
\setmainlanguage[numerals=Devanagari]{bengali}
\setmainlanguage{bengali}
\setotherlanguage{english}
\newfontfamily\bengalifont[Script=Bengali]{NikoshLightBAN}
\usepackage[a4paper,left=0.1cm,right=0.1cm,top=0cm,bottom=0.1cm]{geometry}
\newcommand{\aline}{\\\hline \arabic{theyflines} &&&\rule{0cm}{0.8cm}}
\newcounter{magicrownumbers}
\newcommand\rownumber{\stepcounter{magicrownumbers}\arabic{magicrownumbers}}
\title{রমযানের জোড়, ২০১৯}
\date{}
\begin{document}
\maketitle
\begin{center}
\end{center}
\pagenumbering{gobble}
\newcounter{theyflines}
\noindent
\bgroup
\def\arraystretch{2.6}
\begin{tabular}{@{}l|ll|p{15cm}}
\toprule
 &  নাম &  পরিমাণ & আপনি দিতে ইচ্ছুক (+ চিহ্ন দিয়ে শুধু পরিমাণ লিখুন, আর অপর পৃষ্টায় নাম লিখুন )\\
\toprule
\rownumber & পোলাও এর চাল(১০০টাকা/কেজি) & ১০০ কেজি &\\
\hline
\rownumber & \pbox{5cm}{গরুর গোস্ত(৫০০টাকা/কেজি)\\(সবার পর এইটা দেই)} & ১০০ কেজি &\\
\hline
\rownumber & তেল(১০০টাকা/কেজি) & ১৫ কেজি &\\
\hline
\rownumber & ঘি(৯০০টাকা/কেজি) & ২ কেজি &\\
\hline
\rownumber & দুধ(৯০ টাকা/কেজি) & ১ কেজি &\\
\hline
\rownumber & টক দই & ৫ কেজি & \\ 
\hline
\rownumber & আলু বোখারা & ২ কেজি &\\
\hline
\rownumber & কিসমিস & ২ কেজি &\\
\hline
\rownumber & পেঁয়াজ & ১০ কেজি &\\
\hline
\rownumber & আদা & ২ কেজি &\\
\hline
\rownumber & রসুন & ২ কেজি &\\
\hline
\rownumber & লবণ & ৫ কেজি &\\
\hline
\rownumber & কাঁচা মরিচ & ২ কেজি &\\
\hline
\rownumber & বিরিয়ানি মশলা(Shan) & ৪ প্যাকেট &\\
\hline
\rownumber & মশলা & ৩০০ টাকার & \\
\hline
\rownumber & জায়ফল যতি & ১০০ টাকা &\\
\hline
\rownumber & তেজপাতা & ১০০ গ্রাম & \\
\bottomrule
\end{tabular}
\egroup
\newpage
\noindent
\begin{tabular}{@{}p{0.25cm}|p{5.5cm}|p{1.5cm}|p{13cm}|}
\toprule
& নাম & রুম নং & আইটেমের নাম ও পরিমাণ
\forloop{theyflines}{1}{\value{theyflines} < 31}{\aline}\\
\toprule
\end{tabular}
\end{document}
