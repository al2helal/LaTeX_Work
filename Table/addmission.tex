\documentclass[landscape]{article}
\usepackage{forloop}
\usepackage{polyglossia}
\usepackage{booktabs}
\setmainlanguage[numerals=Devanagari]{bengali}
\setmainlanguage{bengali}
\setotherlanguage{english}
\newfontfamily\bengalifont[Script=Bengali]{NikoshLightBAN}
\usepackage[a4paper,left=0.1cm,right=0.1cm,top=0.1cm,bottom=0.1cm]{geometry}
\newcommand{\aline}{\\\hline \arabic{theyflines} &&&&&&&&&&&\rule{0cm}{1cm}}
\renewcommand{\arraystretch}{2}
\begin{document}
\begin{center}
বিভিন্ন বিশ্ববিদ্যালয়ে ভর্তি ইচ্ছুক ভাইদের নামের তালিকা
\end{center}
\pagenumbering{gobble}
\newcounter{theyflines}
\begin{tabular}{|p{0.13cm}|p{4cm}|p{3cm}|p{2cm}|p{1cm}|p{2cm}|p{3.5cm}|p{4cm}|p{1.5cm}|p{1cm}|p{1cm}|p{1cm}|}
\hline
{\tiny  ক্রমিক} নং &  নাম & মোবাইল নং & জেলা & পূর্বে লাগানো সময়(দিন) & বর্তমান নিয়ত & কোথায় কোথায় পরীক্ষা দিতে ইচ্ছুক & যার কাছে আছে তার নাম & সম্পর্ক & রুম নং & ডিপার্টমেন্ট & বর্ষ \\
\hline
01 & ভাই মোঃ আব্দুল্লাহ & 01763492651 & কুড়িগ্রাম & ৪০ & পরীক্ষার পর ১ চিল্লা & \selectlanguage{english}{DU,JU,JNU,CU} & ভাই মোঃ আতিক & বড় ভাই & ১৬১০  & \selectlanguage{english}{EEE} & \selectlanguage{english}{MS}\\
\hline
\end{tabular}\\

উপরের  নমুনা অনুসারে লিখলে ভাল হয়\\ 

%\vspace*{1cm}
\noindent
\begin{tabular}{|p{0.13cm}|p{4cm}|p{3cm}|p{2cm}|p{1cm}|p{2cm}|p{3.5cm}|p{4cm}|p{1.5cm}|p{1cm}|p{1cm}|p{1cm}|}
\hline
{\tiny  ক্রমিক} নং &  নাম & মোবাইল নং & জেলা & পূর্বে লাগানো সময়(দিন) & বর্তমান নিয়ত & কোথায় কোথায় পরীক্ষা দিতে ইচ্ছুক & যার কাছে আছে তার নাম & সম্পর্ক & রুম নং & ডিপার্টমেন্ট & বর্ষ \\
\hline
\forloop{theyflines}{1}{\value{theyflines} < 11}{\aline}\\
\hline
\end{tabular}
\end{document}
