\documentclass[landscape, legalpaper, 12pt]{article}
\usepackage{forloop}
\usepackage{polyglossia}
\usepackage{booktabs}
\usepackage{dingbat}
\setmainlanguage[numerals=Devanagari]{bengali}
\setmainlanguage{bengali}
\setotherlanguage{english}
\newfontfamily\bengalifont[Script=Bengali]{NikoshLightBAN}
\usepackage[legalpaper, left=0.1cm, right=0.1cm, top=0.1cm, bottom=0.1cm]{geometry}
\newcommand{\aline}{\\\hline \arabic{theyflines} & ভাই মোঃ &&&&&&&ভাই মোঃ&&&&&&&&&\rule{0cm}{1cm}}
\begin{document}
\begin{center}
\selectlanguage{english}


\selectlanguage{bengali}
%তাকাজা ১ঃ \hspace{4cm} ২ঃ \hspace{4cm} ৩ঃ \hspace{4cm} ৪ঃ \hspace{4cm} ৫ঃ \hspace{4cm} ৬ঃ \hspace{4cm}\\ 
তাকাজা ১ঃ ইজতেমার দাওয়াত ২ঃ ইজতেমার ওসুল ৩ঃ জোড়ের দাওয়াত ৪ঃ আজাইম পুরা করতেছে কিনা ৫ঃ তালিমে জুড়তেছে কিনা ৬ঃ \hspace{4cm}\\ 
\end{center}
\pagenumbering{gobble}
\newcounter{theyflines}
\noindent
\begin{tabular}{|p{0.13cm}|p{5cm}|p{1.7cm}|p{3cm}|p{2cm}|p{1cm}|p{2cm}|p{1cm}|p{5cm}|p{1cm}|p{2cm}|p{1.9cm}|p{0.35cm}|p{0.35cm}|p{0.35cm}|p{0.35cm}|p{0.35cm}|p{0.35cm}|}
\hline
{\tiny  ক্রমিক} নং &  নাম &  রুম নং & মোবাইল নং &  ডিপার্টমেন্ট &  বর্ষ &  জেলা & পুঃ লাঃ সময় &  জিম্মাদার সাথীর নাম & আজাইম & সময় দিবে & নিয়ত & ১ & ২ & ৩ & ৪ & ৫ & ৬
\forloop{theyflines}{1}{\value{theyflines} < 14}{\aline}\\
\hline
\end{tabular}

\vspace*{1cm}
নিচের নমুনা অনুসারে লিখলে ভাল হয়\\ 

\noindent
\begin{tabular}{|p{0.13cm}|p{5cm}|p{1.7cm}|p{3cm}|p{2cm}|p{1cm}|p{2cm}|p{1cm}|p{5cm}|p{1cm}|p{2cm}|p{1.9cm}|p{0.35cm}|p{0.35cm}|p{0.35cm}|p{0.35cm}|p{0.35cm}|p{0.35cm}|}
\hline
{\tiny  ক্রমিক} নং &  নাম &  রুম নং & মোবাইল নং &  ডিপার্টমেন্ট &  বর্ষ &  জেলা & পুঃ লাঃ সময় &  জিম্মাদার সাথীর নাম & আজাইম & সময় দিবে & নিয়ত & ১ & ২ & ৩ & ৪ & ৫ & ৬\\
\hline
01 & ভাই মোঃ আব্দুল্লাহ & ১৬১০  & \selectlanguage{english}0176349265 &\selectlanguage{english}EEE \selectlanguage{bengali} & ২য় & কুড়িগ্রাম & ১ চি. & ভাই মোঃ আঃ রহিম & ১০জন & ৯.০৫ - ৯.৩৫ & ৩ দিন & \checkmark & \checkmark & \checkmark & \checkmark & \checkmark & \checkmark\\
\hline
\end{tabular}
\end{document}
