\documentclass{article}
\usepackage{multido}
\usepackage{booktabs}
\usepackage{graphics}
\usepackage{enumitem}
\usepackage[a4paper,left=0.3cm,top=0.5cm,bottom=0.2cm,right=0.3cm]{geometry}
\usepackage{polyglossia}
\setmainlanguage[numerals=Devanagari]{bengali}
\setmainlanguage{bengali}
\setotherlanguage{english}
%\newfontfamily\englishfont[Scale=MatchLowercase]{Linux Biolinum O}
\newfontfamily\bengalifont[Script=Bengali]{Akaash}

\usepackage{pgffor}
\newcommand{\ListItem}[1]{\underline{\makebox[6cm][l]{#1}}}
\newcommand{\minicontent}{\foreach \x in {প্লেট,সসপেন (ঢাকনাসহ) \hfill সেট,বাটি/বোল,ভাত বাড়ার ছোট প্লেট,বন প্লেট,লবণ দানী,গ্লাস,হামান দিস্তা,চাকু,আলু ছেলার কাটার,মাজনী,ভিম,ক্লিপ,বালতি,জগ,মগ,ডিম ভাজার প্যান,দস্তরখানা,চামচ,বটি,ঘুটনি,নুসনি,দড়ি (রং),বাজারের ব্যাগ,চুলা,সিলিন্ডার,কিতাব  \hfill সেট}{\ListItem{\x}\par\bigskip}}
\begin{document}
\noindent
\multido{}{3}{\begin{minipage}{0.33333333333333333333333\textwidth}
জিম্মাদারঃ \\
\newline
সাথীসংখ্যাঃ\\
\newline
\minicontent
\begin{itemize}[itemsep=0.0pt,leftmargin=*]
\item সামানার হেফাজত ঈমানের হেফাজত
\item সামানা গুছিয়ে রাখাও দাওয়াত
\item ইস্তেমায়ী সামানার হেফাজতে ইস্তেমায়ীয়াত  রক্ষা হয়
\item জামাত থেকে এসে তালিকা জমা দিলে ভাল হয়
\end{itemize}
\end{minipage}
}
\end{document}
