\documentclass{article}
\usepackage{forloop}
\usepackage{polyglossia}
\usepackage{booktabs}
\setmainlanguage[numerals=Devanagari]{bengali}
\setmainlanguage{bengali}
\setotherlanguage{english}
\newfontfamily\bengalifont[Script=Bengali]{Akaash}
\usepackage[a4paper,left=0.1cm,right=0.1cm,top=0.1cm,bottom=0.1cm]{geometry}
\newcommand{\aline}{\\\hline \arabic{theyflines} &&&&&&&\rule{0cm}{1cm}}
\begin{document}
\begin{center}
  Extension Building(South Building)\\
যাদেরকে আমরা এখন পর্যন্ত ৩ দিনের জন্য বের করতে পারি নাই, তাদের নামের তালিকা 
\end{center}
\pagenumbering{gobble}
\newcounter{theyflines}
\noindent
\begin{tabular}{|p{0.13cm}|p{5.2cm}|p{1.5cm}|p{1.3cm}|p{1cm}|p{1.4cm}|p{4cm}|p{2.9cm}|}
\hline
{\tiny  ক্রমিক} নং &  নাম &  রুম নং &  ডিপার্টমেন্ট &  বর্ষ &  সেশন &  মোবাইল নং &  জেলা
\forloop{theyflines}{1}{\value{theyflines} < 22}{\aline}\\
\hline
\end{tabular}

\vspace*{1cm}
নিচের নমুনা অনুসারে লিখলে ভাল হয়\\ 

\noindent
\begin{tabular}{|p{0.13cm}|p{5.2cm}|p{1.5cm}|p{1.3cm}|p{1cm}|p{1.4cm}|p{4cm}|p{2.9cm}|}
\hline
{\tiny  ক্রমিক} নং &  নাম &  রুম নং &  ডিপার্টমেন্ট &  বর্ষ &  সেশন &  মোবাইল নং &  জেলা\\
\hline
01 & ভাই মোঃ আব্দুল্লাহ & ১৬১০  & EEE & ২য় & ১৬-১৭ & 01763492651 & কুড়িগ্রাম \\
\hline
\end{tabular}
\end{document}
