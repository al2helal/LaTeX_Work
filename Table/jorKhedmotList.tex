\documentclass[12pt]{article}
\usepackage{forloop}
\usepackage{polyglossia}
\usepackage{booktabs}
\setmainlanguage[numerals=Devanagari]{bengali}
\setmainlanguage{bengali}
\setotherlanguage{english}
\newfontfamily\bengalifont[Script=Bengali]{NikoshLightBAN}
\usepackage[a4paper,left=0.1cm,right=0.1cm,top=0cm,bottom=0.1cm]{geometry}
\newcommand{\aline}{\\\hline \arabic{theyflines} &&&\rule{0cm}{0.8cm}}
\begin{document}
\begin{center}
%Extension Building - 1\\
%যাদেরকে আমরা এখন পর্যন্ত ৩ দিনের জন্য বের করতে পারি নাই, তাদের নামের তালিকা 
\end{center}
\pagenumbering{gobble}
\newcounter{theyflines}
\noindent
\bgroup
\def\arraystretch{2.4}
\begin{tabular}{@{}lll|p{15cm}}
\toprule
ক্রমিক নং &  নাম &  পরিমাণ & আপনি দিতে ইচ্ছুক (+ চিহ্ন দিয়ে শুধু পরিমাণ লিখুন, আর অপর পৃষ্টায় নাম লিখুন )\\
\toprule
০১ & চাল & ৩০ কেজি &\\
\hline
০২ & গরু & ৩০ কেজি &\\
\hline
০৩ & তেল & ৮ কেজি &\\
\hline
০৪ & পেঁয়াজ & ৫ কেজি &\\
\hline
০৫ & ঘি & ৫০০ গ্রাম &\\
\hline
০৬ & আলু বোখারা & ৫০০ গ্রাম &\\
\hline
০৭ & কিসমিস & ৫০০ গ্রাম &\\
\hline
০৮ & দুধ & ৫০০ গ্রাম &\\
\hline
০৯ & বিরিয়ানি মশলা & ৬ প্যাকেট &\\
\hline
১০ & টক দই & ২ কেজি &\\
\hline
১১ & কাঁচা মরিচ & ১ কেজি &\\
\hline
১২ & আদা & ১ কেজি &\\
\hline
১৩ & রসুন & ৫০০ গ্রাম &\\
\hline
১৪ & লবণ & ১.৫ কেজি &\\
\hline
১৫ & জায়ফল যতি & ৫০ টাকা &\\
\bottomrule
\end{tabular}
\hrule
\hrule
\hrule
\vspace{0.5cm}
\textbf{পায়েসের জন্য}\\
\noindent
\begin{tabular}{@{}lll|p{15cm}}
\toprule
ক্রমিক নং &  নাম &  পরিমাণ & আপনি দিতে ইচ্ছুক (+ চিহ্ন দিয়ে শুধু পরিমাণ লিখুন, আর অপর পৃষ্টায় নাম লিখুন )\\
\toprule
০১ & দুধ & ১০ কেজি & \\
\hline
০২ & চাল & ৩০ কেজি & \\
\hline
০৩ & চিনি & ৬ কেজি & \\
\hline
০৪ & গরম মশলা & ৫০ টাকা & \\
\hline
০৫ & চিনাবাদাম & ৫০০ গ্রাম & \\
\bottomrule
\end{tabular}
\egroup
\newpage
\noindent
\begin{tabular}{|p{6cm}|p{3cm}|p{5cm}|p{4cm}|}
\toprule
নাম & রুম নং & আইটেমের নাম & পরিমাণ
\forloop{theyflines}{1}{\value{theyflines} < 31}{\aline}\\
\toprule
\end{tabular}
\end{document}
