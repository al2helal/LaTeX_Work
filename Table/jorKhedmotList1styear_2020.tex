\documentclass[12pt]{article}
\usepackage{forloop}
\usepackage{polyglossia}
\usepackage{booktabs}
\setmainlanguage[numerals=Devanagari]{bengali}
\setmainlanguage{bengali}
\setotherlanguage{english}
\newfontfamily\bengalifont[Script=Bengali]{NikoshLightBAN}
\usepackage[a4paper,left=0.1cm,right=0.1cm,top=0cm,bottom=0.1cm]{geometry}
\newcommand{\aline}{\\\hline \arabic{theyflines} &&&\rule{0cm}{0.99cm}}
\begin{document}
\begin{center}
%Extension Building - 1\\
%যাদেরকে আমরা এখন পর্যন্ত ৩ দিনের জন্য বের করতে পারি নাই, তাদের নামের তালিকা 
\end{center}
\pagenumbering{gobble}
\newcounter{theyflines}
\noindent
\bgroup
\def\arraystretch{2.5}
\begin{center}প্রথম বর্ষের ছাত্রদের জোড় - ২০২০ । সম্ভাব্য জুড়নেওয়ালী প্রথম বর্ষ + সাথীঃ ১৪০ জন\end{center}
\begin{tabular}{@{}lll|p{15cm}}
\toprule
ক্রমিক নং &  নাম &  পরিমাণ & আপনি দিতে ইচ্ছুক (+ চিহ্ন দিয়ে শুধু পরিমাণ লিখুন, আর অপর পৃষ্টায় নাম লিখুন )\\
\toprule
০১ & চাল & ২২ কেজি &\\
\hline
০২ & গরু & ১৫ কেজি &\\
\hline
০৩ & তেল & ৫ কেজি &\\
\hline
০৪ & পেঁয়াজ & ৩ কেজি (বড়) &\\
\hline
০৫ & ঘি & ২৫০ গ্রাম &\\
\hline
০৬ & আলু বোখারা & ৪০০ গ্রাম &\\
\hline
০৭ & কিসমিস & ৪০০ গ্রাম &\\
%\hline
%০৮ & দুধ & ৫০০ গ্রাম &\\
\hline
০৮ & বিরিয়ানি মশলা & ৩ প্যাকেট &\\
\hline
০৯ & টক দই & ২ কেজি &\\
\hline
১০ & কাঁচা মরিচ & ১/২ কেজি &\\
\hline
১১ & আদা & ১/২ কেজি &\\
\hline
১২ & রসুন & ১/২ কেজি &\\
\hline
১৩ & লবণ & ১ কেজি &\\
\hline
১৪ & মশলা & ১০০ টাকার &\\
\hline
১৫ & শসা & ৫ কেজি &\\
%\hline
%১৫ & জায়ফল যতি & ৫০ টাকা &\\
\bottomrule
\end{tabular}
\egroup
\newpage
\noindent
\begin{tabular}{|p{5cm}|p{2cm}|p{7cm}|p{5cm}|}
\toprule
নাম & রুম নং & আইটেমের নাম(একের অধিক হলে কমা(,) দিয়ে লিখেন) & পরিমাণ
\forloop{theyflines}{1}{\value{theyflines} < 26}{\aline}\\
\toprule
\end{tabular}
\end{document}
