\documentclass[landscape]{article}
\usepackage{booktabs}
\usepackage[a4paper,left=0.5cm,top=3cm,bottom=1cm,right=0.5cm]{geometry}
\usepackage{polyglossia}
\setmainlanguage[numerals=Devanagari]{bengali}
\setmainlanguage{bengali}
\setotherlanguage{english}
%\newfontfamily\englishfont[Scale=MatchLowercase]{Linux Biolinum O}
\newfontfamily\bengalifont[Script=Bengali]{Akaash}
\renewcommand{\arraystretch}{2}
\begin{document}
\begin{tabular}{@{}l|p{8cm}|p{8cm}|p{8cm}}
\toprule
নজম & ১ম দিন & ২য় দিন & ৩য় দিন\\
\hline  
দাওয়াত & & &\\
\hline
খেদমত & & &\\
\hline
সকালের তালিম & & &\\
\hline
বিকালের তালিম  & & &\\
\hline
যোহর বাদ কিতাব & & &\\
\hline
এলান & & &\\
\hline
গাস্তের আদব & & &\\
\hline
ঈমানী কথা & & &\\
\hline
মাগরিব বাদ বয়ান  & & &\\
\hline
এশা বাদ কিতাব & & &\\
\hline
মোজাকারা & & &\\
\hline
ফজর বাদ কথা & & &\\
\hline
মসজিদ সাপাই & & &\\
\hline
সামানা পাহারা & & &\\
\hline
প্রথম মজলিসের কথা & & &\\
\bottomrule
\end{tabular}
\begin{itemize}
\item প্রত্যেকেই নিজ নিজ আমল স্মরণ রাখব ইনশাআল্লাহ। 
\end{itemize}
\end{document}
